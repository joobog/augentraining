\documentclass[a4paper,10pt,twoside,notitlepage,twocolumn]{article}

\usepackage{ucs}
\usepackage[utf8]{inputenc}
\usepackage{amsmath}
\usepackage{amssymb}
\usepackage[USenglish]{babel}
\usepackage{fontenc}
\usepackage{graphicx}
\usepackage{hyperref}

% \usepackage[dvips]{hyperref}

\author{Eugen Betke; Gabriel Michalski}
\title{Wiederherstellung der Sehkraft}
\date{09-06-2014}

\begin{document}
\maketitle
\section{Einführung}
Augentraining ist eine sehr umstrittene Methode zur Wiederherstellung der Sehkraft bzw. Heilung der Augen. Im Internet gibt es einerseits viele Erfolgsberichte und andererseits konnten viele Menschen damit kein Erfolg erzielen. Warum? Haben sie vielleicht unterschiedlich trainiert, obwohl sie versucht haben das gleiche Problem zu beseitigen? Oder haben sie mit den gleichen Trainingsmethoden versucht unterschiedliche Probleme zu beseitigen? Oder unterscheiden sich vielleicht menschliche Augen biologisch so weit, dass man bei gleichen Problemen gleiche Trainingsmethoden unterschiedlich auswirken? 

\section{Vorbeitung}
ToDo: Seetafel, Lautschrift, Seetest

\section{Methodik}
Die Vermessung der Dioptrie- und Astigmatismuswerte lassen wir beim gleichen Optiker, an der selben Apparatur und unter den gleichen Bedingungen durchführen. Dies liefert uns am Ende ein zuverlässiges und repräsentatives Ergebnis.

In der zwischen Zeit führen wir unsere eigene Messungen an der Golovin-Sivtsev Tabelle durch und notieren das Datum, die Uhrzeit und die V-Werte für das linke und das Rechte Auge. An diesen Messwerten kann später sehen wie sich die Sehkraft während des Training ändert.

Das Übungsprogramm entnehmen wir der Videovorlesung \cite{lection} und dem dazugehörigen Transskript \cite{transscript}.

ToDo: Beschreibung der Uebungen

ToDo: gesunde Lebensweise: kein Alkohol, Sport, mindestens Trennkost
ToDo: Beschreibung der Person

Das Experiment ist fuer die Dauer von 6 Wochen angesetzt.

\section{Trainingsphase}

\section{Ergebnisse}
\begin{table}
	\begin{tabular}{llll}
	\hline
	Datum & Uhrzeit & linkes Auge & rechtes Auge \\
	\hline
	\end{tabular}
\end{table}

\section{Schluss}

\bibliography{literatur}
\bibliographystyle{alpha}
\end{document}
