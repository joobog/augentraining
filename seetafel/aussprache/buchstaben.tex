\documentclass[a4paper,10pt]{article}
\usepackage{ucs}
\usepackage[utf8]{inputenc}
\usepackage{amsmath}
\usepackage{amssymb}
\usepackage{a4wide}
\usepackage[landscape]{geometry}
%\usepackage[USenglish]{babel

\usepackage[russian,ngerman]{babel}
\usepackage{fontenc}
\usepackage{graphicx}
\usepackage[OT2,T1]{fontenc}
\usepackage[dvips]{hyperref}

\author{Eugen Betke}
\date{08/15/14}

\begin{document}
\begin{table}
	\begin{tabular}{|l | l | l | l | l|}
\hline
Buchstabe							& Russischer Name 	& Aussprache 											& Lautschrift 	& Beispiele\\
\hline
\foreignlanguage{russian}{Ш ш}		& scha				& sch (immer hart) - wie in 'Schule' 					& sch 			& \foreignlanguage{russian}{школа}\\
\foreignlanguage{russian}{Ы ы}		& y					& wird wie i mit zurückgezogener Zunge ausgesprochen.	& y				& \foreignlanguage{russian}{ты}\\
\foreignlanguage{russian}{М м} 		& em				& m - hart wie in 'Mann' oder weich wie in 'Miete'		& m				& Hart: \foreignlanguage{russian}{масло}; Weich: \foreignlanguage{russian}{мясо}\\
\foreignlanguage{russian}{К к} 		& ka				& k - hart wie in 'Kabel' oder weich wie in 'Kiel'		& k				& Hart: \foreignlanguage{russian}{кот}; Weich: \foreignlanguage{russian}{кит}\\
\foreignlanguage{russian}{И и} 		& i					& i - wie in '\foreignlanguage{russian}{Kilo}'; nach \foreignlanguage{russian}{ж, ш, ц} - wie \foreignlanguage{russian}{ы}	& i				& \foreignlanguage{russian}{один}; wie \foreignlanguage{russian}{'ы'}:  \foreignlanguage{russian}{машина}\\
\foreignlanguage{russian}{Б б} 		& be				& b - hart wie in 'Buch' oder weich wie in 'Birke'		& b				& Hart: \foreignlanguage{russian}{быть}; Weich: \foreignlanguage{russian}{бить}\\
\foreignlanguage{russian}{Н н}		& en				& n - hart wie in 'Nord' oder weich wie in 'nötig'		& n	 			& Hart: \foreignlanguage{russian}{ныть}; Weich: \foreignlanguage{russian}{нить}\\
\hline
	\end{tabular}


\end{table}
	
\end{document}
